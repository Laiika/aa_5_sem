\chapter{Аналитическая часть}

\section{Расстояние Левенштейна}

Расстояние Левенштейна -- это минимальное количество операций вставки, удаления и замены символа, необходимое для преобразования одной символьной последовательности в другую.

Для решения задачи поиска расстояния Левенштейна нужно найти последовательность операций с минимальной суммарной ценой. Цена каждой операции:
\begin{itemize}[left=\parindent]
	\item $w(a, b) = 1$ -- для замены одного символа на другой;
	\item $w(a, \lambda) = 1$ -- для удаления одного символа;
	\item $w(\lambda, b) = 1$ -- для вставки одного символа;
	\item $w(a, a) = 0$ -- для двух совпавших символов.
\end{itemize}

Пусть $s_1$ и $s_2$ -- две строки над некоторым алфавитом, тогда расстояние Левенштейна можно подсчитать по следующей рекуррентной формуле (\ref{eq1.1})
\begin{equation}
	\label{eq1.1}
	D(i, j) = \begin{cases}
		0, ~~\text{$i = 0, j = 0$}\\
		i, ~~\text{$j = 0, i > 0$}\\
		j, ~~\text{$i = 0, j > 0$}\\
		\min \lbrace \\
		\qquad D(i, j - 1) + 1,\\
		\qquad D(i - 1, j) + 1,\\
		\qquad D(i - 1, j - 1) + \\
		\qquad \begin{cases}
			0, ~~\text{если $s_1[i] = s_2[j]$}\\
			1, ~~\text{иначе}
		\end{cases}\\
		\rbrace, ~~~\text{$i > 0, j > 0$}
	\end{cases}
\end{equation}

\section{Расстояние Дамерау~---~Левенштейна}

Расстояние Дамерау~---~Левенштейна -- это минимальное количество операций вставки, удаления, замены и перестановки соседних символов, необходимое для преобразования одной символьной последовательности в другую.

Алгоритм поиска расстояния Дамерау~---~Левенштейна аналогичен алгоритму поиска расстояния Левенштейна. Дополнительно необходимо учесть операцию перестановки соседних символов, цена которой равна 1. Тогда расстояние Дамерау~---~Левенштейна можно подсчитать по формуле (\ref{eq1.2})
\begin{equation}
	\label{eq1.2}
	D(i, j) = \begin{cases}
		0, ~~\text{$i = 0, j = 0$}\\
		i, ~~\text{$j = 0, i > 0$}\\
		j, ~~\text{$i = 0, j > 0$}\\
		\min \lbrace \\
		\qquad D(i, j - 1) + 1,\\
		\qquad D(i - 1, j) + 1,\\
		\qquad D(i - 1, j - 1) + \\
		\qquad \begin{cases}
			0, ~~\text{если $s_1[i] = s_2[j]$}\\
			1, ~~\text{иначе}
		\end{cases}\\
		\qquad,\\
		\qquad \begin{cases}
			D(i - 2, j - 2) + 1, \\
			\text{если $i, j > 1, s_1[i] = s_2[j - 1], s_2[j] = s_1[i - 1]$}\\
			\infty, ~~\text{иначе}
		\end{cases}\\
		\rbrace, ~~~\text{$i > 0, j > 0$}
	\end{cases}
\end{equation}

\section*{Вывод из аналитической части}

В данном разделе были рассмотрены алгоритмы вычисления расстояний Левенштейна и Дамерау~---~Левенштейна. Формулы вычисления расстояний являются рекуррентными, поэтому каждый из алгоритмов может быть реализован как с использованием рекурсии, так и без. 
