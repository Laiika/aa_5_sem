\chapter*{Заключение}
\addcontentsline{toc}{chapter}{Заключение}

В результате выполнения лабораторной работы при исследовании алгоритмов нахождения расстояний Левенштейна и Дамерау~---~Левенштейна были изучены и отработаны навыки динамического программирования.

В ходе выполения лабораторной работы:

\begin{enumerate}[label={\arabic*)}]
    \item были изучены расстояния Левенштейна и Дамерау~---~Левенштейна;
    \item были разработаны алгоритмы нерекурсивного метода поиска расстояния Левенштейна, нерекурсивного метода поиска, рекурсивного метода поиска и рекурсивного с кешированием метода поиска расстояния Дамерау~---~Левенштейна;
    \item был реализован каждый из описанных алгоритмов;
    \item были выбраны инструменты для замера процессорного времени выполнения реализаций алгоритмов;
    \item было проведено сравнение реализованных алгоритмов по времени работы:
  	выявлено, что реализация рекурсивного алгоритма без кеширования уступает по времени всем другим реализациям, нерекурсивная реализация работает в 1.9 раза быстрее рекурсивной реализации с кешем, нерекурсивная реализация алгоритма поиска расстояния Левенштейна работает в 1.2 -- 1.3 раза быстрее нерекурсивной реализации алгоритма нахождения расстояния Дамерау~---~Левенштейна;
    \item было проведено сравнение реализованных алгоритмов по используемой памяти: выявлено, что реализации, использующие матрицу, уступают рекурсивной без кеша.
\end{enumerate}

Таким образом, все поставленные задачи были выполнены, а цель достигнута.
