\chapter*{Введение}
\addcontentsline{toc}{chapter}{Введение}

Расстояние Левенштейна, или редакционное расстояние, -- это минимальное количество операций замены, вставки и удаления символа, которое нужно выполнить над одной последовательностью символов, чтобы получить другую. Определение расстояния Дамерау~---~Левенштейна получается добавлением к указанным действиям операции перестановки соседних символов \cite{lev}.

Расстояние Левенштейна используется для исправления ошибок в словах, поиска дубликатов текстов, сравнения геномов и прочих операций с символьными последовательностями.

\textbf{Целью данной работы} является  изучение и исследование особенностей задач динамического программирования на материале алгоритмов нахождения расстояний Левенштейна и Дамерау~---~Левенштейна.

Для достижения поставленной цели необходимо выполнить следующие
задачи:
\begin{enumerate}[label={\arabic*)}]
    \item изучить расстояния Левенштейна и Дамерау~---~Левенштейна;
    \item разработать алгоритмы нерекурсивного метода поиска расстояния Левенштейна, нерекурсивного метода поиска расстояния Дамерау~---~Левенштейна, рекурсивного метода поиска расстояния Дамерау~---~Левенштейна, рекурсивного с кешированием метода поиска расстояния Дамерау~---~Левенштейна;
    \item реализовать разработанные алгоритмы;
    \item выбрать инструменты для замера процессорного времени выполнения реализаций алгоритмов;
    \item сравнить алгоритмы по процессорному времени работы реализаций;
    \item сравнить алгоритмы по используемой памяти.
\end{enumerate}
