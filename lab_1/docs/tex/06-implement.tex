\chapter{Технологическая часть}

\section{Средства реализации}

Для реализации данной лабораторной работы был выбран язык C++ \cite{c++}. В его стандартной библиотеке содержится класс wstring, работающий и с латиницей, и с кириллицей \cite{wstring}. В качестве среды разработки был выбран текстовый редактор VS Code. Замеры процессорного времени проводились при помощи функции clock из библиотеки time.h \cite{clock}.

\section{Реализации алгоритмов}

В данном подразделе представлены листинги кода ранее описанных алгоритмов:
\begin{itemize}[left=\parindent]
    \item алгоритм нерекурсивного метода поиска расстояния Левенштейна (листинг \ref{lst:lev_iter});
    \item алгоритм нерекурсивного метода поиска расстояния Дамерау~---~Левенштейна (листинг \ref{lst:dl_iter});
    \item алгоритм рекурсивного метода поиска расстояния Дамерау~---~Левенштейна (листинг \ref{lst:dl_rec});
    \item алгоритм рекурсивного метода поиска расстояния Дамерау~---~Левенштейна с кешированием  (листинги \ref{lst:dl_rec_cache}).
\end{itemize}

\mylisting{Реализация алгоритма нерекурсивного метода поиска расстояния Левенштейна}{lev_iter}{12-45}{algs.cpp}
\mylisting{Реализация алгоритма нерекурсивного метода поиска расстояния Дамерау~---~Левенштейна}{dl_iter}{48-86}{algs.cpp}
\mylisting{Реализация алгоритма рекурсивного метода поиска расстояния Дамерау~---~Левенштейна}{dl_rec}{89-117}{algs.cpp}
\mylisting{Реализация алгоритма рекурсивного с кешированием метода поиска расстояния Дамерау~---~Левенштейна}{dl_rec_cache}{120-161}{algs.cpp}

\section{Описание тестирования}

В таблице \ref{tab:tests} приведены тесты для алгоритмов поиска расстояний Левенштейна и Дамерау~---~Левенштейна.

\begin{table}[h!]
	\begin{center}
    \begin{threeparttable}
        \captionsetup{justification=raggedright,singlelinecheck=off}
        \caption{\label{tab:tests}Тесты}
        \begin{tabular}{|c|c|c|c|}
			\hline
			\textbf{Слово №1} & \textbf{Слово №2} & \textbf{Ожидаемый результат} & \textbf{Результат} \\ [0.8ex] 
			\hline
			s & c & 1 1 1 1 & 1 1 1 1\\
			\hline
			hell & hallo & 2 2 2 2 & 2 2 2 2\\
			\hline
			corn & cron & 2 1 1 1 & 2 1 1 1\\
			\hline
			honda & hyundai & 3 3 3 3 & 3 3 3 3\\
			\hline
			sun & sun & 0 0 0 0 & 0 0 0 0\\
			\hline
			qwer & wqre & 3 2 2 2  & 3 2 2 2\\
			\hline
			йцук & цйку & 3 2 2 2  & 3 2 2 2\\
			\hline
		\end{tabular}
    \end{threeparttable} 
	\end{center}
\end{table}

\section*{Вывод из технологической части}

В данном разделе был выбран инструмент для замера процессорного времени, были реализованы алгоритмы поиска расстояний Левенштейна и Дамерау~---~Левенштейна. Также было проведено тестирование реализованных алгоритмов.
