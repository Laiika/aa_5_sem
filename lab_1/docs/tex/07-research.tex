\chapter{Исследовательская часть}

\section{Технические характеристики}

Технические характеристики устройства, на котором выполнялись замеры времени:

\begin{itemize}
	\item операционная система Windows 11 64--bit;
	\item оперативная память 16 ГБ;
	\item процессор 2.40 ГГц Intel Core i5--1135G7 \cite{intel}.
\end{itemize}

\section{Постановка эксперимента по замеру времени}

Для оценки процессорного времени работы реализаций алгоритмов поиска расстояний Левенштейна и Дамерау~---~Левенштейна был проведен эксперимент, в котором определялось влияние длины символьных последовательностей на время работы каждого из алгоритмов. Тестирование проводилось на словах длиной от 5 до 10 символов с шагом 1 и от 20 до 100 с шагом 10. На последовательностях длиной от 20 символов замеры для рекурсивного алгоритма без кеширования не выполнялись из-за быстрого увеличения времени работы. Поскольку методы замера процессорного времени имеют погрешность и возвращают для достаточно коротких задач константу 0, каждый алгоритм запускался по 500 раз, и для полученных 500 значений определялось среднее арифметическое, которое заносилось в таблицу результатов.

Результаты эксперимента были представлены в виде таблицы и графиков, приведенных
в следующем подразделе.

\clearpage

\section{Результаты эксперимента}

\begin{table} [h!]
	\caption{Таблица времени работы реализаций алгоритмов (в мкс)}
	\begin{center}
		\begin{tabular}{|c c c c c|}
			\hline
			Длина слова & Лев. & Д.-Л. итер. & Д.-Л. рек. & Д.-Л. рек. кеш  \\ [0.5ex]
			\hline
			5 & 0 & 0 & 0 & 0\\
			\hline
			6 & 0 & 0 & 60.80 & 0\\
			\hline
			7 & 0 & 0 & 319.94 & 0\\
			\hline
			8 & 0 & 0 & 1807.34 & 0\\
			\hline
			9 & 0 & 0 & 9831.58 & 0\\
			\hline
			10 & 0 & 12.57 & 55262.60 & 0\\
			\hline 
			20 & 0 & 0 & -- & 0\\
			\hline
			30 & 30.00 & 32.21 & -- & 32.80\\
			\hline
			40 & 32.04 & 35.33 & -- & 64.04\\
			\hline
			50 & 34.57 & 41.03 & -- & 96.28\\
			\hline
			60 & 62.36 & 75.76 & -- & 128.47\\
			\hline
			70 & 62.58 & 94.60 & -- & 188.06\\
			\hline 
			80 & 94.50 & 126.73 & -- & 282.39\\
			\hline
			90 & 126.47 & 172.12 & -- & 312.43\\
			\hline
			100 & 156.01 & 190.34 & -- & 360.07\\
			\hline
		\end{tabular}
	\end{center}
\end{table}
\img{60mm}{img01}{Сравнение времени работы реализаций нерекурсивных алгоритмов поиска расстояний Левенштейна и Дамерау~---~Левенштейна}{first}

\img{60mm}{img02}{Сравнение времени работы реализаций нерекурсивного алгоритма поиска расстояния Дамерау~---~Левенштейна и рекурсивного алгоритма поиска с кешем}{second}

\img{60mm}{img03}{Сравнение времени работы реализаций всех алгоритмов на последовательностях малой длины}{third}

\clearpage

\img{60mm}{img04}{Сравнение времени работы реализаций всех алгоритмов, кроме рекурсивного, на последовательностях средней и большой длины}{fourth}

Можно отметить, что на символьных последовательностях длиной до 30 символов реализация нерекурсивного алгоритма поиска расстояния Левенштейна и реализация нерекурсивного алгоритма поиска расстояния Дамерау~---~Левенштейна отрабатывают приблизительно за одинаковое время. Однако при увеличении длины последовательностей реализация алгоритма поиска расстояния Левенштейна становится более эффективной по времени. Она работает быстрее в 1.2 -- 1.3 раза.

Нерекурсивная реализация алгоритма поиска расстояния Дамерау~---~Левенштейна и рекурсивная реализация с кешем работают в разы быстрее рекурсивной без кеша. Но на той же длине слов нерекурсивная реализация работает примерно в 1.9 раза быстрее рекурсивной реализации с кешем.

\section{Расчет используемой памяти}

Алгоритмы нахождения расстояний Левенштейна и Дамерау~---~Левенштейна не отличаются друг от друга с точки зрения используемой памяти.

Для каждого вызова рекурсивной реализации алгоритма Дамерау~---~Левенштейна выделяется память под:
\begin{itemize}[left=\parindent]
	\item 2 строки;
	\item длины строк;
	\item локальная переменная;
	\item возвращаемое значение;
	\item адрес возврата.
\end{itemize}

Максимальное количество вызовов равно сумме длин двух строк.

Максимальное значение выделяемой памяти выражается формулой (\ref{eq4.1})
\begin{equation}\label{eq4.1}
	mem1 = (4 \cdot size(int) + 2 \cdot size(string)) \cdot (|s_1| + |s_2|)
\end{equation}

Для единственного вызова нерекурсивной реализации алгоритма Дамерау~---~Левенштейна выделяется память под:
\begin{itemize}[left=\parindent]
	\item 2 строки;
	\item длины строк;
	\item матрицу размерами, равными длинам строк, увеличинным на единицу;
	\item 4 локальных переменных;
	\item возвращаемое значение;
	\item адрес возврата.
\end{itemize}

Максимальное значение выделяемой памяти выражается формулой (\ref{eq4.2})
\begin{equation}\label{eq4.2}
	mem1 = 7 \cdot size(int) + 2 \cdot size(string) + size(int ) \cdot (|s_1| + 1) \cdot (|s_2| + 1)
\end{equation}

Память, используемая рекурсивной реализацией, растет пропорционально сумме длин строк, в то время как память, используемая нерекурсивной реализацией, растет пропорционально произведению длин строк, то есть по выделяемой памяти нерекурсивная реализация проигрывает рекурсивной. В рекурсивной реализации с кешем память так же растет пропорционально произведению длин строк.

\section*{Вывод из исследовательской части}

По времени выполнения нерекурсивная реализация и реализация с кешем гораздо эффективнее  рекурсивной без кеша. При этом нерекурсивная реализация работает в 1.9 раза быстрее рекурсивной реализации с кешем. Нерекурсивная реализация алгоритма поиска расстояния Левенштейна работает в 1.2 -- 1.3 раза быстрее нерекурсивной реализации алгоритма нахождения расстояния Дамерау~---~Левенштейна.

По выделяемой памяти реализации, использующие матрицы, проигрывают рекурсивной без кеша: максимальный размер выделяемой памяти в них пропорционален произведению длин строк, в то время как в рекурсивной без кеша -- сумме длин строк.
